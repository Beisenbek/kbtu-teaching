\documentclass{beamer}
    %
    % Choose how your presentation looks.
    %
    % For more themes, color themes and font themes, see:
    % http://deic.uab.es/~iblanes/beamer_gallery/index_by_theme.html
    %
    \mode<presentation>
    {
      \usetheme{default}      % or try Darmstadt, Madrid, Warsaw, ...
      \usecolortheme{default} % or try albatross, beaver, crane, ...
      \usefonttheme{default}  % or try serif, structurebold, ...
      \setbeamertemplate{navigation symbols}{}
      \setbeamertemplate{caption}[numbered]
    } 


    \usepackage{bookmark}
    \usepackage[english]{babel}
    \usepackage[utf8x]{inputenc}
    \usepackage{csvsimple}
    \usepackage{listings} %For code in appendix
    \lstset
    { %Formatting for code in appendix
        language=c++,
        basicstyle=\footnotesize,
        numbers=left,
        stepnumber=1,
        showstringspaces=false,
        tabsize=1,
        breaklines=true,
        breakatwhitespace=false,
    }
    
    \title[Programming Principles I]{Lecture I}
    \author{Baisakov B., Akshabaev A., Mukhsimbaev B., Amanov A., Buzaubakov R.}
    \institute{Kazakh-British Technical University}
    \date{Week 1}

    \begin{document}

    \begin{frame}
      \titlepage
    \end{frame}
    
    \begin{frame}{Outline}
      \tableofcontents
    \end{frame}
    
    \section{Introduction to code structure}
    \begin{frame}{Introduction to code structure}
        \lstinputlisting{samplecode1.cpp}
    \end{frame}

    \section{Compiling and executing program}
    \begin{frame}{Compiling and executing program}
      1. g++ file.cpp \textit{//compiling}\\
      2. for mac and linux: ./a.out \textit{//executing}\\
      3. for windows: a.exe \textit{//executing}\\
    \end{frame}

    \section{Introduction to data types}
    \begin{frame}{Introduction to data types}
      \begin{tabular}{ |c | c|  }
        \hline
        Type & Size \\
        \hline
        bool, char, unsigned char, signed char  &	1 byte \\
        short, unsigned short  &	2 bytes \\
        float, int, unsigned int, long, unsigned long & 4 bytes \\
        double, long double, long long &	8 bytes \\
        \hline
       \end{tabular}
    \end{frame}

    \section{Primitive data types: int, double, float}
    \begin{frame}{Representing Numbers: int, double, float}
      int a;\\
      double b;\\
      float c;\\
    \end{frame}

    \section{Comments}
    \begin{frame}{Comments}
      \lstinputlisting{samplecode2.cpp}
    \end{frame}
       \section{Introduction to piazza}
    \begin{frame}{Introduction to piazza}
      \url{https://bit.ly/2wtGLFa}\\
        or\\
      \url{piazza.com/kazak_british_technical_university/fall2018/pp1}\\
      access code: \textbf{pp1}\\
    \end{frame}

    \section{Laboratory work 1}
    \begin{frame}{Laboratory work 1}
    \end{frame}
      details in piazza\\
    \end{document}